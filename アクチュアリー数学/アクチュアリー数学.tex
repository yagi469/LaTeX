\documentclass[upLaTeX,dvipdfmx,b5j,openany,autodetect-engine,dvipdfmx-if-dvi,ja=standard,11pt]{jsbook}

\usepackage{geometry}
\usepackage{amsmath}
\usepackage{amssymb,otf,tcolorbox}
\usepackage{fancyhdr,enumitem,ascmac,multicol,wrapfig}
\usepackage{newtxmath,newtxtext}
\usepackage{tabularx}
\usepackage{graphicx}
\usepackage{amssymb,otf,tcolorbox}
\usepackage{titlesec,titletoc,color,thmbox,multicol,ulem}
\usepackage{pgf,tikz,pgfplots}

\newcolumntype{C}{>{\centering\arraybackslash}X}
\usetikzlibrary{shapes.callouts}
\usetikzlibrary{arrows}

\renewcommand{\chaptermark}[1]{\markright{\thesection \hspace{1em}#1}}
\renewcommand{\baselinestretch}{1.1}

\pagestyle{myheadings}
%ページスタイル設定(全ページ編集指定)

\markboth{}{}
%ヘッダーの設定

\renewcommand{\prechaptername}{第}
\renewcommand{\postchaptername}{講}
%チャプター名の設定

\newcommand{\Frac}[2]{\dfrac{\,#1\,}{\,#2\,}}
%線の長い分数

\newcounter{toi}[section]
\newcommand{\toi}{\refstepcounter{toi}\hangindent=2em\noindent {\bf \large 問題 \thechapter-\thetoi} \indent}
%問題コマンド

\newcommand{\etoi}{\refstepcounter{toi}\hangindent=2em\noindent {\bf \large 演習問題 \thechapter-\thetoi} \indent}
%演習問題コマンド

\newcommand{\htoi}{\refstepcounter{toi}\hangindent=2em\noindent {\bf \large 必修問題 \thechapter-\thetoi} \indent}
%必修問題コマンド

\newcommand{\rtoi}{\refstepcounter{toi}\hangindent=2em\noindent {\bf \large 復習問題 \thechapter-\thetoi} \indent}
%復習問題コマンド

\newcounter{rei}[section]
\newcommand{\rei}{\refstepcounter{rei}\hangindent=1em\noindent{\bf 例題\therei}\hspace{1em}}
%例題コマンド

\newcommand{\ans}{\fbox{{\bf 解答}}}
%解答記号

\newcommand{\hon}[1]{}
%出典問題集コマンド・問題タイプ等 テキスト本体には掲載しないがメモとして残しておきたい内容

\setlist[enumerate,1]{label=(\arabic*),topsep=0pt,parsep=2pt,partopsep=0pt,itemsep=0pt,leftmargin=3em,labelsep=1em}
%enumerate第1レベルの設定

\newcommand{\dan}[2]{\begin{multicols}{#1} #2 \end{multicols}}
\multicolsep = 3pt
%段組み短縮コマンド

\newcommand{\op}[1]{
\begin{thmbox}[M,leftmargin=3\zw,thickness=0.8pt]{\colorbox[gray]{0.85}{\textgt{ワンポイント}}}
#1
\end{thmbox}
}
%ワンポイントコマンド

\newcommand{\cm}[1]{
\begin{thmbox}[M,leftmargin=3\zw,thickness=0.8pt]{\colorbox[gray]{0.85}{\textgt{コメント}}}
#1
\end{thmbox}
}
%コメントコマンド

\newcommand{\source}[2]{
\begin{flushright}
出典:{#1}年度~{#2}
\end{flushright}
}
%出典コマンド

\newcommand{\note}{
\newpage
\begin{center}
\Large{N~~O~~T~~E}
\end{center}
\newpage
}
%NOTEコマンド

%===========================================================
%プリアンブルおわり
%===========================================================

\begin{document}

%===========================================================
%ここからが表紙
%===========================================================

\newgeometry{top=1cm,left=5mm,right=1cm,bottom=1cm}

\thispagestyle{empty}

\hrulefill%最上部のライン

\vspace{5mm}

\hspace*{10mm}\Large{2021年度 1学期}%開講年と講座種別

\hspace*{3mm}\dotfill%ドットライン(左3mm空ける)

\hspace*{5mm}\Huge{{\bf 【講座名】}}%ここが講座名
\begin{flushright}
\LARGE{}%副題があればここ
\end{flushright}

\vfill

\begin{flushright}
\Large{{\bf 【予備校名】}}
\end{flushright}

\hrulefill%最下部のライン

\normalsize%フォントサイズのリセット
\restoregeometry%余白設定のリセット

\newpage
\thispagestyle{empty}

\newgeometry{twoside,top=20truemm,bottom=25truemm,left=30truemm,right=20truemm}

~
\newpage

\setcounter{page}{1}

%===========================================================
%ここまでが表紙と後処理
%===========================================================

%内容はここから


\chapter*{はじめに}

【ここに「はじめに」の内容】

\vspace{10mm}

\begin{flushright}
2021年8月 テキスト作成 【作成者名や作成団体などを入力します】
\end{flushright}

\newpage

\chapter*{講義の進め方とテキストの構成}

【ここに講座を行うにあたって書いておきたいことを入力します】

\vfill

\noindent
{\bf テキストで使う記号}

\begin{tabular}{cl}
$\ast$ & 比較的難しい問題や難しい事項。主に補充問題で使用している。\\
$\clubsuit$ & 大学入試共通テストでも重要になる内容。センター試験で頻出だった内容。\\
$\Re$ & 既に学習した事項。忘れている場合は前に戻って復習しよう。\\
$\dagger$ & やや難しい内容。難関大入試で問われることが多く、意欲のある生徒向けの内容。\\
$\ddagger$ & 大学の数学の内容だが理解に役立つ内容。完全に理解する必要はない。
\end{tabular}

\newpage

%目次============================================
\titlecontents{chapter}[0pt]{}{\thecontentslabel \hspace{5mm}}{}{\titlerule*{・}\makebox[2mm][r]{\thecontentspage}}
%目次タイトルの設定 titletoc

\titlecontents{section}[0pt]{}{\thecontentslabel \hspace{5mm}}{}{\titlerule*{・}\makebox[2mm][r]{\thecontentspage}}
%目次タイトルの設定 titletoc

\tableofcontents

%目次は{n}でn=0でchapter階層まで
%============================================

\newpage

%ここからが内容

\chapter{【各講のタイトルはchapter階層で書く】}

\toi
\hon{剰余で分類}
$7^{n}$ を 6 で割った余りは 1 であることを証明したい.
\begin{enumerate}
\item 数学的帰納法を用いて証明せよ.
\item 二項定理を用いて証明せよ.
\end{enumerate}
\source{2020}{豊橋技科大・前期}

\ans

\vfill

\toi
\hon{領域の問題}
放物線 $y=x^{2}+a x+b$ により, $x y$ 平面を 2つの領域に分割する.
\begin{enumerate}
\item 点 (-1,4)と点 (2, 8) が放物線上にはなく別々の領域に属するような $a,~b$の条件を求めよ。更に, その条件を満たす$(a,~b)$ の領域を $a b$ 平面に図示せよ.
\item $a$, bが (1)で求めた条件を満たすとき, $a^{2}+b^{2}$ がとり得る値の範囲を求めよ.
\end{enumerate}
\source{2015}{愛知教育大}

\vfill
\newpage

\toi
\hon{面積計算}
$a,~b$ を定数とし,実数
\[f(x)=\int_{0}^{x}\left(t^{2}+a t+b\right) dt\]
が$\displaystyle x=-\frac{1}{3}$および$x=1$で極値をとるものとする.~このとき,~次の問に答えよ.
\begin{enumerate}
\item 定数 $a$ の値を答えよ.
\item 関数$f(x)$の極小値を答えよ.
\item 関数$f(x)$ の極大値を答えよ.
\item $m$ が(2)における極小値であるとき,~曲線$y=f(x)$ と直線 $y=m$によって囲まれた部分の面積を答えよ.
\end{enumerate}
\source{2020}{防衛大・理工}

\vfill

\note

\end{document}