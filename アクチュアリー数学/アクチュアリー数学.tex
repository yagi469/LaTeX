\documentclass{jsarticle}
\begin{document}
\textbf{問題1.} 次の(1)~(12)の各問について、空欄に当てはまる最も適切なものをそれぞれの選択肢の中から1つ選び、解答用紙の所定の欄にマークしなさい。なお、同じ選択肢を複数回選択してもよい。
\rightline{各5点(計60点)}

\noindent{(1)} 外見から区別のつかない2つの箱がある。1つの箱Rには9個の赤玉と6個の白玉が入っており、もう1つの箱Wには6個の赤玉と9個の白玉が入っている。2つの箱から1つを無作為に選び、その箱から一度に5個同時に玉を取り出したところ、赤玉が3個、白玉が2個であった。このとき、選ばれた箱がRである確率は\fbox{     }である。\\

\textbf{解答} \\


箱Rが選ばれる事象を$R$、箱Wが選ばれる事象を$W$とする。また、箱から一度に5個同時に玉を取り出した結果をFとする。求める確率$P(R|F)$はベイズの公式により次式で計算される。

\begin{equation}
P(R|F) = \frac{P(F|R)\cdot P(R)}{P(F|R)\cdot P(R) + P(F|W)\cdot P(W)}
\end{equation}

2つの箱から1つの箱を無作為に選ぶので、

\begin{equation}
\end{equation}

\begin{equation}
\end{equation}

\begin{equation}
\end{equation}

\begin{equation}
\end{equation}

\begin{equation}
\end{equation}

\end{document}