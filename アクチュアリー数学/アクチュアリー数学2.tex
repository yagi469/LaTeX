\documentclass{jsbook}
\usepackage{amsmath}

\newcounter{toi}[section]
\newcommand{\toi}{\refstepcounter{toi}\hangindent=2em\noindent {\bf \large 問題 \thesection-\thetoi}.\indent}
%問題コマンド

\newcommand{\ans}{\fbox{{\bf 解答}}\\}
%解答記号

\newcommand{\hon}[1]{}
%出典問題集コマンド・問題タイプ等 テキスト本体には掲載しないがメモとして残しておきたい内容

\renewcommand{\labelenumi}{(\arabic{enumi})}
%enumerate第1レベルの設定

\newcommand{\source}[2]{
\begin{flushright}
出典:{#1}年度~{#2}
\end{flushright}
}
%出典コマンド

\setlength{\textwidth}{\fullwidth}  %本文の幅(textwidth)を全体の幅(=ヘッダ部の幅)にそろえる
\setlength{\evensidemargin}{\oddsidemargin} %偶数ページの余白と奇数ページの余白をそろえる

%===========================================================
%プリアンブルおわり
%===========================================================

\begin{document}

\tableofcontents

\chapter{確率分野問題}
\section{離散一様分布}
\newpage
\section{二項分布}
\newpage
\section{ポアソン分布}
\newpage
\section{幾何分布、ファーストサクセス分布}
\newpage
\section{負の二項分布}
\newpage
\section{超幾何分布}
\newpage
\section{多項分布}
\toi あるパン工場で製造されるロールパンの重さ$Z$(g)は25.0gから27.0gまでの一様分布に従うとし、ロールパンの個数を表す確率変数$X_1,X_2,X_3$をそれぞれ次で定める。
\begin{itemize}
\item 重さが
\item
\end{itemize}
\newpage
\section{正規分布}
\newpage

\section{指数分布、ガンマ分布}
\toi ある携帯通信キャリアの通信利用料金月額は、月内の通信利用料が20(GB)以内であれば定額2,000円、20(GB)を超えると500円が定額料金に加算され、それ以降1(GB)増えるごとに500円がさらに加算されていくという。携帯通信利用者の月内の通信利用料$X$(GB)が平均4(GB)の指数分布に従うとき、通信利用料金月額の期待値に最も近い数値は\fbox{     }円である。なお、必要であれば、$e=2.718$を用いてよい。
\newpage
\section{ベータ分布、多元データ分布}
\newpage
\section{トレーズ、確率変数の分解}
\newpage
\section{加法定理、複合分布}
\newpage
\section{ベイズの定理}

\toi 外見から区別のつかない2つの箱がある。1つの箱Rには9個の赤玉と6個の白玉が入っており、もう1つの箱Wには6個の赤玉と9個の白玉が入っている。2つの箱から1つを無作為に選び、その箱から一度に5個同時に玉を取り出したところ、赤玉が3個、白玉が2個であった。このとき、選ばれた箱がRである確率は\fbox{     }である。
\source{2021}{過去問}

\ans

箱Rが選ばれる事象を$R$、箱Wが選ばれる事象を$W$とする。また、箱から一度に5個同時に玉を取り出した結果をFとする。求める確率$P(R|F)$はベイズの公式により次式で計算される。

\begin{equation}
P(R|F) = \frac{P(F|R)\cdot P(R)}{P(F|R)\cdot P(R) + P(F|W)\cdot P(W)}
\end{equation}

2つの箱から1つの箱を無作為に選ぶので、

\begin{equation}
P(R) = P(W) = \frac{1}{2}
\end{equation}

$P(F|R)$は、箱Rから一度に5個同時に玉を取り出すとき、赤玉が3個、白玉が2個となる確率であるから、

\begin{equation}
P(F|R) = \frac{\binom{9}{3}\cdot\binom{6}{2}}{\binom{15}{5}}
\end{equation}

$P(F|W)$は、箱Wから一度に5個同時に玉を取り出すとき、赤玉が3個、白玉が2個となる確率であるから

\begin{equation}
P(F|W) = \frac{\binom{6}{3}\cdot\binom{9}{2}}{\binom{15}{5}}
\end{equation}

よって、求める確率は、
\begin{equation}
\begin{split}
P(R|F)& = \frac{\frac{\binom{9}{3}\cdot\binom{6}{2}}{\binom{15}{5}}\cdot(\frac{1}{2})}{\frac{\binom{9}{3}\cdot\binom{6}{2}}{\binom{15}{5}}\cdot(\frac{1}{2}) + \frac{\binom{6}{3}\cdot\binom{9}{2}}{\binom{15}{5}}\cdot(\frac{1}{2})} \\
&=\frac{\binom{9}{3}\cdot\binom{6}{2}}{\binom{9}{3}\cdot\binom{6}{2} + \binom{6}{3}\cdot\binom{9}{2}} \\
&=\frac{(\frac{9\cdot8\cdot7}{3\cdot2})\cdot(\frac{6\cdot5}{2})}{(\frac{9\cdot8\cdot7}{3\cdot2})\cdot(\frac{6\cdot5}{2}) + (\frac{6\cdot5\cdot4}{3\cdot2})\cdot(\frac{9\cdot8}{2})} \\
&=\frac{7}{7 + 4} = \frac{7}{11}
\end{split}
\end{equation}

\newpage
\section{漸化式}
\newpage
\section{一様分布の変換、一様分布の和・差・積・商、三角分布}
\newpage
\section{正規分布の変換、対数正規分布}
\newpage
\section{コーシー分布、t分布、F分布、第2種パレート分布}

\chapter{統計分野問題}
\section{中心極限定理、チェビシェフの不等式}
\newpage
\section{順序統計量系}
\newpage
\section{点推定系}
\newpage
\section{区間推定系}
\newpage
\section{有限母集団}
\newpage
\section{検定系}

\chapter{モデリング分野問題}
\section{回帰分析系}
\newpage
\section{時系列解析系}
\newpage
\section{確率過程系}
\newpage
\section{シミュレーション系}
\begin{equation}
\end{equation}

\begin{equation}
\end{equation}

\begin{equation}
\end{equation}
\end{document}