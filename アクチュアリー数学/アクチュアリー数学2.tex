\documentclass{jsbook}
\usepackage{amsmath}

\newcounter{toi}[section]
\newcommand{\toi}{\refstepcounter{toi}\hangindent=2em\noindent {\bf \large 問題 \thechapter-\thetoi}.\indent}
%問題コマンド

\newcommand{\ans}{\fbox{{\bf 解答}}\\}
%解答記号

\newcommand{\hon}[1]{}
%出典問題集コマンド・問題タイプ等 テキスト本体には掲載しないがメモとして残しておきたい内容

\renewcommand{\labelenumi}{(\arabic{enumi})}
%enumerate第1レベルの設定

\newcommand{\source}[2]{
\begin{flushright}
出典:{#1}年度~{#2}
\end{flushright}
}
%出典コマンド

\setlength{\textwidth}{\fullwidth}  %本文の幅(textwidth)を全体の幅(=ヘッダ部の幅)にそろえる
\setlength{\evensidemargin}{\oddsidemargin} %偶数ページの余白と奇数ページの余白をそろえる

%===========================================================
%プリアンブルおわり
%===========================================================

\begin{document}

\tableofcontents

\chapter{2021年度 過去問}

\toi 次の(1)~(12)の各問について、空欄に当てはまる最も適切なものをそれぞれの選択肢の中から1つ選び、解答用紙の所定の欄にマークしなさい。なお、同じ選択肢を複数回選択してもよい。
\begin{flushright}各5点(計60点)\end{flushright}

\begin{enumerate}
\item 外見から区別のつかない2つの箱がある。1つの箱Rには9個の赤玉と6個の白玉が入っており、もう1つの箱Wには6個の赤玉と9個の白玉が入っている。2つの箱から1つを無作為に選び、その箱から一度に5個同時に玉を取り出したところ、赤玉が3個、白玉が2個であった。このとき、選ばれた箱がRである確率は\fbox{     }である。\\

\ans


箱Rが選ばれる事象を$R$、箱Wが選ばれる事象を$W$とする。また、箱から一度に5個同時に玉を取り出した結果をFとする。求める確率$P(R|F)$はベイズの公式により次式で計算される。

\begin{equation}
P(R|F) = \frac{P(F|R)\cdot P(R)}{P(F|R)\cdot P(R) + P(F|W)\cdot P(W)}
\end{equation}

2つの箱から1つの箱を無作為に選ぶので、

\begin{equation}
P(R) = P(W) = \frac{1}{2}
\end{equation}

$P(F|R)$は、箱Rから一度に5個同時に玉を取り出すとき、赤玉が3個、白玉が2個となる確率であるから、

\begin{equation}
P(F|R) = \frac{\binom{9}{3}\cdot\binom{6}{2}}{\binom{15}{5}}
\end{equation}

\begin{equation}
\end{equation}

\begin{equation}
\end{equation}

\begin{equation}
\end{equation}
\end{enumerate}
\end{document}